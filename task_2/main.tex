\documentclass[10pt]{article}
\usepackage[total={170mm,230mm}]{geometry}

\usepackage{cmap}
\usepackage{hyperref}
\usepackage[utf8]{inputenc}
\usepackage[T2A]{fontenc}
\usepackage[russian]{babel}

\usepackage{graphicx}
\usepackage{xcolor}
\usepackage{amssymb}
\usepackage{amsfonts}
\usepackage{amsmath}
\usepackage{amsthm}
\usepackage{physics}
\usepackage{wrapfig}
\usepackage{cancel}
\usepackage{pdfpages}
\usepackage{hyperref}
% \usepackage{bibtex}

\title{Домашнее задание №1}
\author{Александр Козлов}
\date{\today}

\begin{document}

\maketitle

\section*{Формулировка задания}

Дан гамильтониан одномерной квантово-механической системы с потенциалом в виде гауссовой ямы
\begin{equation}
    H = - \dv[2]{}{x} + V_0\, e^{-x^2},
\end{equation}
где $V_0 < 0$. Требуется сделать следующее.

\begin{enumerate}
    \item Найти константы связи $V_0$, при которых в системе возникает одно, два и три связанных состояния.

    \item Исследовать зависимость вычислительных затрат от размера сетки.

    \item Исследовать зависимость погрешности энергий состояний от размера сетки и границ бокса.
\end{enumerate}

Задания следует выполнять методом коллокации с разложением по набору функций в пространстве кубических сплайнов $S_{3,2}$.

\section{Численное решение}

Рассматриваемое уравнение Шрёдингера (УШ) имеет вид
\begin{equation}
    -\dv[2]{\psi}{x} + V_0\, e^{-x^2}\, \psi  = E\, \psi
    \label{eq:SE}
\end{equation}
или
\begin{equation}
    \dv[2]{\psi}{x}  + (E - V_0\, e^{-x^2})\, \psi  = 0.
\end{equation}
Стоит заметить, что такое уравнение соответствует обычному одномерному УШ при $\hbar=1$ и $m=1/2$.

Прежде всего зададим равномерную сетку
\begin{equation}
    x_0 = -R,\; x_1 = x_0 + \delta,\; x_2 = x_0 + 2\delta,\; \ldots,\; x_k = x_0 + k\delta,\; \ldots,\; x_M = x_0 + M \delta = R
\end{equation}
с шагом $\delta = 2R/M$, где $M$~---~целое положительное число, а $R$~---~положительное действительное число.

Решать уравнение будем методом, основанным на разложении волновой функции по некоторому набору функций $S_l(x)$
\begin{equation}
    \psi(x) = \sum\limits_{l} f_l S_l(x).
    \label{eq:decomp}
\end{equation}

\subsection{Набор функций}

Для каждой $k$-ой точки стеки определена пара функций
\begin{equation}
    \begin{split}
        \Phi_{k}(x) = \qty{-\dfrac{1}{\delta^2}(x-x_k)^2 + 1}\cdot \theta(x>x_{k-1}) \cdot \theta(x<x_{k+1}),\\
        \Psi_{k}(x) = \qty{- \dfrac{1}{\delta^2}(x-x_k)^3 + (x-x_k)} \cdot \theta(x>x_{k-1}) \cdot \theta(x<x_{k+1}).\\
    \end{split}
\end{equation}
Из таких функций нужно собирать набор $S_l(x)$, по которому будем раскладывать нашу волновую функцию. Набор функций можно ввести следующим образом:
\begin{equation}
    S_l(x) =
    \begin{cases}
        \alpha_1 \Psi_0(x) + \alpha_2 \Phi_0(x), &l=1;\\
        \beta_1 \Psi_M(x) + \beta_2 \Phi_M(x), &l=2M;\\
         \Phi_{(l-1)/2}(x), &{l\,\textrm{mod}\,2 = 1\textrm{ и }l\ne 1};\\
        \Psi_{l/2}(x), &{l\,\textrm{mod}\,2 = 0\textrm{ и }l\ne 2M}
    \end{cases}
    ,\quad l = \overline{1, 2M}.
\end{equation}
Параметры $\alpha$ и $\beta$ выбираются на основе поставленного граничного условия. Для определённости поставим нулевое граничное условие
\begin{equation}
    \psi(x_0) = \psi(x_M) = 0,
\end{equation}
тогда параметры $\alpha$ и $\beta$ можно выбрать такими:
\begin{equation}
    \alpha_2=\beta_2=0,\quad \alpha_1=\beta_1=1.
\end{equation}

\subsection{Точки коллокации}

В качестве точек коллокации будут использованы нули второго полинома Лежанра на отрезке $\qty[-1,\;1]$, отображённые с этого отрезка на отрезки сетки $\qty[x_{k-1},\;x_{k}],\; k=\overline{1,M}$. Кратко точки коллокации определить можно следующим образом:
\begin{equation}
    x^{(c)}_t =
    \begin{cases}
        x_{(t-1)/2} + \dfrac{\delta}{2} \qty(1 - \dfrac{1}{\sqrt{3}}), &{t\,\textrm{mod}\,2 = 1};\\
        x_{t/2} - \dfrac{\delta}{2} \qty(1 - \dfrac{1}{\sqrt{3}}), &{t\,\textrm{mod}\,2 = 0}
    \end{cases}
    ,\quad t = \overline{1, 2M}.
\end{equation}

\subsection{Проекция уравнения Шрёдингера на дельта-функции}

Подставляем разложение волновой функции по кубическим сплайнам \eqref{eq:decomp} в уравнения Шрёдингера \eqref{eq:SE} и проецируем полученное уравнение на дельта-функции $\delta(x-x^{(c)}_t)$, приходим к соотношению
\begin{equation}
    \sum\limits_{l=1}^{2M} f_{l}
    \mel**{\delta(x-x^{(c)}_t)}{H-E}{S_{l}(x)}=0,\quad l=\overline{1,2M}.
\end{equation}
Функции $S_l(x)$ мы выбрали локализованными, это позволяет заметно сократить число слагаемых в уравнении, так как дельта-функция с центром в точке $x^{(c)}\in\qty[x_{k-1},\;x_{k}]$ даёт ненулевые интегралы лишь для тех $S_l(x)$, которые отличны от нуля на интервале $\qty[x_{k-1},\;x_{k}]$. Однако, для записи этого вывода требуется учитывать чётность $t$. Нетрудно получить, что для нечетного индекса $t$ выживают слагаемые с индексом $l=t-1,\;t,\;t+1,\;t+2$, а для четного $t$ выживают слагаемые с индексом $l=t-2,\;t-1,\;t,\;t+1$.

Таким образом, у нас получается $2M$ уравнений на $2M$ неизвестных коэффициентов разложения $\qty{f_l}_{l=1}^{2M}$ и на 1 неизвестную $E$, того $2M+1$ неизвестных. То есть допускается неединственность решения. Что разумно, ведь в дискретном спектре может быть несколько состояний.

Чтобы решить такую систему, надо посчитать определитель квадратной матрицы
\begin{equation}
    \hat A(E) = \qty(\mel**{\delta(x-x^{(c)}_t)}{H-E}{S_{l}(x)})_{l,t}
\end{equation}
и приравнять его к нулю, тем самым получив характеристическое уравнение на допустимые уровни энергии $E$. Отсюда следует, что такой метод позволяет разрешить лишь $2M$ связных состояний.

\subsection{Нахождение корней характеристического уравнения}

Как было показано выше, в итоге задача отыскания уровней энергий дискретного спектра сводится к решению характеристического уравнения
\begin{equation}
    f(E) = \det\qty(\hat A(E)) = 0
\end{equation}
в интервале значений $E \in \qty(-\abs{V_0},\; 0)$. Это можно сделать методом секущих, для чего надо пройти с малым шагом по энергии рассматриваемый диапазон энергий; шаг определяется тем, насколько точно мы хотим найти ответ.

\end{document}
